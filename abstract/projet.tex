\documentclass[11pt]{article}
\usepackage[utf8]{inputenc}
\usepackage[francais]{babel}
\usepackage[T1]{fontenc}
\begin{document}

\begin{titlepage} % Suppresses displaying the page number on the title page and the subsequent page counts as page 1
	\newcommand{\HRule}{\rule{\linewidth}{0.5mm}} % Defines a new command for horizontal lines, change thickness here
	
	\center % Centre everything on the page
	
	%------------------------------------------------
	%	Headings
	%------------------------------------------------
	
	\textsc{\LARGE UQAC}\\[1.5cm] % Main heading such as the name of your university/college
	
	\textsc{\Large 8INF257 -- informatique mobile}\\[0.5cm] % Major heading such as course name
	
	%% \textsc{\large Minor Heading}\\[0.5cm] % Minor heading such as course title
	
	%------------------------------------------------
	%	Title
	%------------------------------------------------
	
	\HRule\\[0.4cm]
	
	{\huge\bfseries Boite à outils}\\[0.4cm] % Title of your document
	
	\HRule\\[1.5cm]

	{\large\textit{Auteurs}}\\
	Baptise \textsc{Boismorand}\\
	Joël \textsc{Costamagna}\\ 
    Nathan \textsc{Voizeux} \\
	Alpha Kabinet \textsc{Keita}\\
	
	
	

	\vfill% Position the date 3/4 down the remaining page
	
	{\large \today} 

\end{titlepage}

\section{présentation du projet}
	Cette application permettra de regrouper différents outils pour permettre un suivi des activités sportives. L'application présentera différentes informations de la vie quotidienne ainsi que des fonctionnalités spécifiques à la randonnée. L'application permettra également de stocker et d'analyser les données récoltées sur l'activité physique de l'utilisateur.
	
\section{exemple de fonctionnalités à implémenter}
	\begin{itemize}
		\item boussole;
		\item météo;
		\item carte;
		\item suivi du nombre de pas;
		\item suivi du nombre de calories;
		\item suivi du parcours effectué;
		\item analyse des données (graphiques);
		\item création d'itinéraire de randonnée;
	\end{itemize}
	
\section{Étapes de réalisation}
	\begin{enumerate}
		\item choix des fonctionnalités;
		\item maquettes de l'interface graphique;
		\item séparation en taches réalisables (backlog);
		\item mise en place de l'environnement de développement;
		\item développement de l'interface globale;
		\item développement des fonctionnalités;
		\item phase de test;
		\item corrections et debug;
		\item rédaction du rapport;
		\item bilan du projet;
	\end{enumerate}
\end{document}













































